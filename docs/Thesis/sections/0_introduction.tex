\chapter{Introduction}
\pagenumbering{arabic}
\setcounter{page}{1}

Human pose estimation aims at detecting the pose or skeleton of a person based on visual information only. It finds many applications, from games to medical applications\cite{kumarapu2020animepose, ClinicalApplicationChen, MedicalAnimation}. However, since human pose estimation is a difficult task, it is prone to errors\cite{HPEIsHard}. These errors can be caused by different factors, such as the environment, the camera, and the person. These errors can cause the joints to be in the incorrect position or missing. This can cause the pose estimation to be incorrect, and therefore, the human-computer interaction will be hampered.

One of the hampering factors for human pose estimation is the user itself. If the posture is not as the model expects then errors might occur. This is especially true for applications that are designed for rehabilitation and exercise purposes. In these applications, the user is often elderly and has limited mobility, as is the case for games developed by SilverFit\footnote{\url{https://www.silverfit.com/en/}}. SilverFit is a serious gaming company that develops games for rehabilitation with a special focus on geriatric patients. In their games, SilverFit uses human pose estimation to detect the pose of the player and use it to control the game to make exercise more enjoyable while promoting activity.

For some exercises designed by SilverFit, human pose estimation is not sufficiently reliable to create an enjoyable experience for patients in every environment. Especially sedentary exercises often cause the human pose estimation to fail or to produce unreliable results.

In this chapter, the research question that is answered in this thesis is discussed. Furthermore, the procedure with which the research question is answered is laid out. Finally, the fundamentals of human pose estimation and the metrics that are used to analyse the results are discussed, and related work is presented.

\section{Research question}

As mentioned earlier, a major problem with human pose estimation is that it is not possible to tell if the joints are faulty or not. This is a problem for SilverFit, as they want to be able to tell if the joints are faulty or not. Using faulty joints can decrease the efficacy of the training effect of the developed games and can make them very frustrating to use and develop. A joint is considered faulty if it is not in the incorrect position, i.e. the distance from the theoretical position is greater than a chosen threshold, or if it missing from the skeleton.

In this thesis, we first ask what problems occur during human pose estimation and what common error sources are. We aim to find which problems are the most common and which joints are most affected by the errors. This will help give an overview of the issues related to human pose estimation and help develop ways to detect these issues. 

Once we know the issues that occur during human pose estimation, we aim to develop a method that can estimate if a joint produced by human pose estimation is potentially faulty. We  

Finally, we 


\section{Process Pipeline}
\label{sec:process_pipeline}

In the scope of the thesis, three steps were defined that are necessary to achieve the goal of the thesis. These steps are the analysis of the problem, the data collection and processing, and the model development.

During the analysis of the problem, different factors are discussed which might influence the human pose estimation. Based on these factors exercises are designed to emulate different scenarios with different difficulties. This is discussed in Section \ref{sec:errors}.

The data collection and processing are discussed in Section \ref{sec:data_acquisition}. The data is captured using a custom tool, FESDData, that was developed for this thesis. The tool allows capturing predefined exercises and labelling them with error labels. The data is then processed and stored in a custom format.

Using the collected dataset a data loader and model are derived. The data loader performs the necessary preprocessing steps and the model is trained to detect faulty joints. This is discussed in Section \ref{sec:model_development}.

Finally, the model is evaluated on a test set and the results are discussed in Section \ref{sec:results}.
\section{Fundamentals}

In this section, REPLACE_WE explain some of the fundamentals that are used during this thesis. 

\subsection{Machine Learning}

\subsection{Evaluation metrics and mathematical formulas}

\subsubsection{Precision and Recall}

\subsubsection{F1-Score}

\subsubsection{Cross-Entropy Loss}
\section{Related Work}
\label{sec:related_work}
% related work before fundamentals
% no subsections but paragraphs
% make it already finished
\subsection{Human Pose Estimation}


\subsubsection{RGB Pose Estimation}

% OpenPose
While OpenPose developed Hand Pose\cite{OpenPoseHand} and also Multi-Person Human Pose Estimation \cite{OpenPoseMulti}, $REPLACE_OUR$ main focus lies on their most recent pose estimator \cite{OpenPosePose} and their CNN network \cite{OpenPoseCNN}. Openpose uses affinity fields. The affinity fields are a set of 2D Gaussian distributions that are used to estimate the pose of a human. The affinity fields are used to estimate the pose of a human by estimating the probability of a joint being in a certain location. The probability of a joint being in a certain location is calculated by summing the probability of the joint being in that location for each of the Gaussian distributions.

\subsubsection{RGBD Pose Estimation}

\subsection{Action Detection}

\subsection{Fault Detection}

Human Pose Estimation using Iterative Error feedback. \cite{IterativeErrorFeedback}

% \textbf{This still needs a lot of work}


% A plethora of methods have been developed to estimate the pose of a human. In this section, $REPLACE_WE$ will discuss some of the methods that have been developed to estimate the pose of a human. Additionally, $REPLACE_WE$ discuss datasets that have been developed to test the performance of the methods. Finally, $REPLACE_WE$ discuss some of the methods that have been developed to estimate the fault.

% \subsection{Human Pose Estimation}


% Human Pose Estimation using Iterative Error feedback. \cite{IterativeErrorFeedback}

% While OpenPose developed Hand Pose\cite{OpenPoseHand} and also Multi-Person Human Pose Estimation \cite{OpenPoseMulti}, $REPLACE_OUR$ main focus lies on their most recent pose estimator \cite{OpenPosePose} and their CNN network \cite{OpenPoseCNN}. Openpose uses affinity fields. The affinity fields are a set of 2D Gaussian distributions that are used to estimate the pose of a human. The affinity fields are used to estimate the pose of a human by estimating the probability of a joint being in a certain location. The probability of a joint being in a certain location is calculated by summing the probability of the joint being in that location for each of the Gaussian distributions.

% \subsubsection{Reviews}


% A review of point cloud-based human pose estimation \cite{ReviewPointcloudHPE}

% A review of 2D human pose estimation methods \cite{ReviewHPE}

% \subsubsection{RGB Pose Estimation}

% But $REPLACE_WE$ wont go into much detail as $REPLACE_WE$ focus on RGBD data.

% The limited number of multi-modal datasets causes the existence of human pose estimators for cameras from different angles to be small. One example of multi-modal human pose estimation was introduced by Jingxiao Zheng et al.\cite{MultiModalHPERGBD}. In their paper 

% \subsubsection{RGBD Pose Estimation}

% \textbf{This is a bit out of context:}
% As mentioned by Jingxiao Zheng et al. in \cite{MultiModalHPERGBD}, the key points or joints of the skeleton do not lay on the surface of the person and therefore the determination of the exact position of the joints are not a direct projection on the depth image or the point cloud.

% \cite{PASCUALHERNANDEZ2022102225}

% \cite{RGBDHPEforRoboticTaskLearning}




% \subsection{RGBD CNNs}

% Early HPE algorithm uses trees \cite{EarlyRGBDHPE}

% CNNs more useful for images and stuff. Cnns are not a new invention yada yada yada \cite{OldCNN}. But like many things in the neural network Biz, they were limited by the hardware available at the time. They have since formed the basis of many new methods in computer vision, such as Human Pose estimation. Especially AlexNet \cite{AlexNet} and VGG \cite{VGG} proved the potential of CNNs in Computer Vision tasks. 

% \subsection{Object Detection}

% \cite{Chen2021} Proposes different methods of fusion for RGBD data.

% \subsubsection{Depth Completion}

% Realsense with Tensorflow \cite{TensorflowRealsense} uses U-Net for depth completion \cite{UNET}.

% \subsubsection{Action Recognition}

% Cool CNN --> \cite{ElboushakiAbdessamad2020MAmf}

% Another Review on human pose estimation but this time it is for action recognition \cite{ReviewHPEforActionRecognition}

% In \cite{Seddik2017} Seddik et al. introduce different fusion methods for action recognition. They use RGB, Depth, and Skeleton data. After detecting the features for each modality, they fuse the features using different methods as can be seen in Figure \ref{fig:fusionmethods}. Seddik et al. use different bags of visual words (BoVW).

% Human3.6M: Large Scale Datasets and Predictive Methods for 3D Human Sensing in Natural Environments \cite{h36m_pami}

% Latent Structured Models for Human Pose Estimation \cite{IonescuSminchisescu11} 
