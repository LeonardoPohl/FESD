\chapter{Introduction}

The code of this thesis is available on \href{https://github.com/LeonardoPohl/FESD}

\section{Research question}

In this thesis, we try to find out if it is possible to develop a method that can tell if a joint produced by human pose estimation is potentially faulty. We try to achieve this by building a dataset using RGBD cameras from different angles.

This project includes both the creation of the dataset as well as the data population with estimated human pose data and the training and evaluation of a model for fault estimation. The dataset contains different challenging scenarios, which make human pose detection more error-prone. These challenges include but are not limited to; lighting, background, clothing and accessories attached to the wrist and ankles, and proximity of the limbs to objects. We found that these have the largest effect on the performance of human pose estimation. 
