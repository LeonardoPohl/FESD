\section{Person}

Finally, one of the main error sources of human pose estimation is the person. The person can cause difficulties in the human pose estimation process by moving, wearing specific clothes, or having a different body posture. Body posture is of special importance for SilverFit since SilverFit specialises in games for rehabilitation and elderly people. Elderly people have different body postures than the average person, which can cause difficulties in the human pose estimation process.

\subsection{Clothes}

\subsection{Training Equipment}

\subsection{Exercises}

Finally, the most important factor is the exercise that is carried out. In this section, we define some exercises that are easy to detect as well as some exercises that are difficult to detect. We also discuss the difficulties that cause the exercises to pose issues for human pose estimation.

\subsubsection{Trivial Exercises}

In the most trivial case, the person is standing still with their arms stretched to the side. In this case, the person is not moving and the joints are not changing position. This is the easiest case for human pose estimation, as the joints are always in the same position. However, this is not a realistic case, as the person is not exercising but it offers a baseline for the other exercises.

The arms stretched to the side reduce the possibility of occlusion, as the side of the body is not blocking the joints of the arms. Furthermore, the person is standing still, which reduces the possibility of occlusion as well.

\subsubsection{Easy Exercises}

\subsubsection{Difficult Exercises}