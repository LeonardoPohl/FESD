\section{Dataset}

We ran multiple iterations of the recording process to find the best possible setup, which reduces the light interference as much as possible and which offers the best results with the resources at hand. We reproduced the setup of the camera as it is used at SilverFit. At SilverFit the camera is mounted at $175cm$. The camera is angled downward at a $70^\circ$ angle. We measure the height with a measuring tape. To estimate the correct angle is very difficult. Thankfully, the Realsense Camera provides us with an accelerometer, which measures the acceleration of the camera. Therefore, it also measures the gravity. An example of the measurements can be seen in Figure \ref{fig:accellerometer}. With these measurements we can calculate the percentage of downward force and know the exact angle at which the camera is to the ground. Additionally, we can fix any roll of the camera providing us with a straight image. 

\begin{equation} \label{eq:orientation}
\dfrac{y}{x+y+z} * 90^{\circ} = Angle
\end{equation}

Equation \ref{eq:orientation} was used to achieve the specified orientation. In our case the target angle in degree is $70^\circ$. Therefore, we know that ${y}/{(x+y+z)} * 90^{\circ} = 70^{\circ}$. Since we eliminated the roll prior we know that $x=0$. Hence, we know that to achieve an angle of 70 degrees, ${y}/{(y+z)}$ has to be equal to ${70^\circ}/{90^\circ}$. The values of $z$ and $y$ can be seen in Figure \ref{fig:accellerometer}. In the Figure $z \approx -3.923 {m}/{s^2}$ and $y \approx -8.855 {m}/{s^2}$, therefore, ${y}/{(y+z)} * 90^\circ = 62.3 \neq 70$ so the camera needs to be rotated further. This process has to be repeated until the angle is correct.

\begin{figure}
  \centering
  \includegraphics[width=0.5\textwidth]{figures/FESD/accelerometer.png}
  \caption[Realsense Accelerometer]{Accelerometer data from the Realsense camera used to set up the camera.}
  \label{fig:accellerometer}
\end{figure}

\subsection{Analysis}

An important aspect of the dataset is the structure and distribution of data and their labels. In total we recorded all 13 exercises, mentioned in Section \ref{sec:exercises}, twice. Each recording session consists of exactly 300 frames.

If a joint cannot be detected by Nuitrack it automatically gets zero coordinates, i.e. every value is zero. This makes it easy to automatically label these joints as faulty, in particular with the error label $1$ - Joint Missing. However, for the rest of the errors each frame has to be manually inspected and each joint considered. Since this requires a lot of work we reduced the labeled frames to 10 percent of the original size. Therefore, each exercise contains 60 frames. 

\subsubsection{Distribution of Errors}

An important factor of how well a model can be trained on data is the balance of the dataset. In this case the dataset is balanced by the error labels. In Figure \ref{fig:statistics_err} we can see the distribution of errors in the dataset as a whole. We see that ...

\begin{figure}
  \centering
  \textit{A histogram depicting the distribution of errors over all exercises (not created yet but probably, mostly No error then wrong position and then not found.)}
  %\includegraphics[width=0.5\textwidth]{}
  \caption[Error Distribution]{The distribution of errors}
  \label{fig:statistics_err}
\end{figure}

To diversify the dataset we recorded different exercises with varying difficulties. In Figure \ref{fig:statistics_err_diff} we see that the difficulty has an influence on the correctness of the joints. This proves the soundness of our design proposals for challenging exercises from Section \ref{sec:exercises}.

\begin{figure}
  \centering
  \textit{A histogram similar to the previous graph. However, With 4 different classes at each error (x). (Not created yet. Based on labelling it seems to be as expected, i.e. Trivial mostly no error and hard mostly wrong.)}
  %\includegraphics[width=0.5\textwidth]{}
  \caption[Error Distribution with Difficulty]{The distribution of errors with additional difficulty distinction.}
  \label{fig:statistics_err_diff}
\end{figure}