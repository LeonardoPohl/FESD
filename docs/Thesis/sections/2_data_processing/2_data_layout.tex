\section{Data layout}

In this section, we discuss the data layout that is used to store the data. This makes it possible to reuse the data in the future for any other application. As mentioned earlier, the data is made up of multiple different streams or modalities. There are two separate visual streams, the RGB stream, and the depth stream, as well as the estimated human pose, the time stamps, and the recording metadata.

The visual streams are normalised and combined into a single file. We use OpenCV to store the RGB and the depth data into a single matrix and after the stream into a single file per frame. The RGB data is normalised to have values between 0 and 1, whereas the depth data is stored in meters. To improve the size of the data as well as the read and write speed, we store the visual data as binary files instead of using the human readable FileStore system provided by OpenCV.

The human pose estimated by Nuitrack as well as the error labels are stored in a separate json file. The separate frames are stored in a list of frames. Each frame contains a list of all people that were detected. Each person contains a list of joints as well as an error label. In some cases, a person might be detected completely wrong, but we do not clean this data as it is still valuable data for model development. Each joint is stored with real-world coordinates which are stored in meters. These real-world coordinates are labeled $x$, $y$, and $z$. Additionally, the 2D projection and depth of the joint are stored in image coordinates and meters for the depth. The 2D projection is labeled $u$ and $v$ and the depth is labeled $d$. For the final model only one dimensionallity is chosen. However, to ensure that not data is lost and that the different models can be trained on different modalities, we store all the data.

The error label is an integer which is the error id specific for joint and skeleton errors. The errors corresponding to the error ids are explained in section \ref{sec:data_labeling}. Finally, we store the timestamp of each frame. This may be useful for future applications.
