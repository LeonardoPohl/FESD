\section{Data Augmentation}
\label{sec:data_augmentation}

Once the data is cleaned and $REPLACE_WE$ filter recordings with too many missing joints, $REPLACE_WE$ can augment the data to simulate a larger amount of data with the ability to create faulty scenarios controlled. One major fault is the seemingly random detachment of the joint to a side. This especially affects the legs and arms, therefore $REPLACE_WE$ will have a bias toward limbs with this augmentation. 

Furthermore, $REPLACE_WE$ randomly move the joints with low confidence. Another fault is the disappearance of joints. $REPLACE_WE$ use the same bias as with the random detachment, i.e. $REPLACE_WE$ take the limb bias, as well as the confidence into consideration.

This phase allows $REPLACE_US$ to create a large amount of data with a controlled amount of faults. This is important since $REPLACE_WE$ want to be able to train a model that can detect faults in the data. The augmented data is stored in a separate file so that $REPLACE_WE$ can use it to train the model and compare it to the original manually checked ground truth.

\textit{\textbf{TODO} Create some screenshots of the augmented data from the different methods.}