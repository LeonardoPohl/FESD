\subsection{Data labeling}
\label{sec:data_labeling}

A large part of the data preparation is the labelling of the data. The data is labelled with error labels. We define two different areas of errors. First, there are skeleton errors. Skeleton errors occur when the pose estimator detects a human in places where there are no humans. This can be caused by the pose estimator detecting a human in the background due to certain features that the estimator assumes are human.

Second, there are joint errors. Joint errors occur when the pose estimator detects a joint in the wrong place. This can be caused by the pose estimator detecting a joint in the background due to certain features that the estimator assumes are a joint. It can also be caused by the estimator labelling a joint incorrectly. 

For example, the estimator might label the left foot as the right foot. This is a common error, especially when the limbs are close too each other. An estimator might also not detect a joint at all. This might be caused through occlusion, be it by another joint, an object, or by the image border. Most applications avoid the last cause for occlusion by defining a minimum distance from the camera and specific camera placement to ensure that the user is always fully in view.

In the data, no error is denoted with $0$. If the whole skeleton is in a wrong position it can be labeled as faulty and subsequently every joint will be labeled with $2$. If a joint is not detected at all, it is labelled with $1$. If a joint is detected in the wrong place that is outside of the body, or somewhere where there should not be a joint, it is labelled with $2$. If a joint is detected in the approximate position of where another joint should be then it is labelled with $3$.

Implicitly, this creates two general labels, either a joint is faulty, i.e. the error label is $1$, $2$, or $3$, or it is not faulty, i.e. the error label is $0$. This makes the task easier, as it is a binary classification rather than a multi-class classification. However, we find the result to be enlightening as they might be more accurate and therefore more reliable.