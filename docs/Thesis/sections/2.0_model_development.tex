\chapter{Model development}

While there could be multiple approaches to fault estimation, we have chosen to use a deep learning approach. The reason for this is that deep learning has shown to be very successful in many different fields, such as image classification, object detection, and image segmentation. The reason for this is that deep learning can learn the features of the data by itself, without the need for manual feature extraction. This is especially useful in our case, as we have a large amount of data, but we do not know which features are important for the fault estimation.

Other possible solutions could be to use rule-based systems, which use inverse kinematics, or use frame-to-frame joint comparison to detect discreptancies, however, these are quite limited and might result in either too many false positives or false negatives. Furthermore, these rules, such as the frame-to-frame joint comparison, are not always applicable to all types of movements, and therefore might not be able to detect all types of faults in all cases.