\section{Stream pre-processing}

To get the best possible results we need to make sure that the cameras are set up in exactly the intended way. 

\subsection{Recording session set-up}

We consider different environmental setups to increase the significance of the results. In total, we achieve $6 * 2 * 2 * 2 * 2 * 2 = 192$ different combinations, which we record $3 times$ for $30s$, resulting in around $17280\s$ of recording which is recorded at around $30 FPS$ resulting in $500000$ Frames in the dataset. We consider the following setups:

\paragraph{Lighting}

RGBD cameras function with infrared light therefore is the lighting of a scene essential. We found that direct sunlight interferes with some RGBD cameras more than others based on the infrared range that is used. Since the exact sunlighting is not controllable we choose to make it as optimal as possible to improve reproducibility. Therefore, we choose a room with no sunlight but we do include artificial light to reduce any damage that might occur to visibility issues.

\paragraph{Relative Camera Position}

At SilverFit, cameras are attached above a screen at a height of $180 \{cm}$ facing downward at around $20\deg$. To form a more general model, we will experiment with different setups and angles. We experiment with six different setups in total. Three setups from different angles ($20\deg, 0\deg, 340\deg$) at two different heights ($180cm$, $120cm$). The different setups can be seen in figure TODO.

\textit{
  \textbf{UNSURE} We Develop a functionality that lets us determine the exact height and orientation of the camera. We do this by detecting the floor and thereby calculating the height of the camera and the angle at which it is pointing downward. We can also detect if the camera is not completely straight and therefore might influence the results.
}

\paragraph{Sitting or standing}

From experience, we know that detecting the joints correctly is influenced by the position of the participant. This is especially true for the difference between a sitting and a standing patient. Human pose detection is in general more reliable if the patient is standing, due to reduced occlusion. We record each scenario sitting and standing.

\paragraph{Clothing and ankle and wrist attachments}

Clothing can have a similar effect on the efficacy of HPE as lighting. If the participant is wearing black pants infrared light will be absorbed rather than reflected leading to 'blind spots' in the legs. Since the legs are already more unreliable than the rest of the body, these blind spots can negatively affect HPE. 

Since SilverFit develops games for rehabilitation, the supervising physiotherapist might choose to attach weights to the ankles and/or wrists to increase the effectiveness of the exercise. We therefore also include attached and held weights to simulate difficult situations.

\paragraph{Background}

The background of the scene can have a significant effect on the results. We, therefore, record the same scenario with and without a visible background, i.e. a wall is behind the participant or there is no wall within the maximum sensor range ($6m$).

\paragraph{Crampedness of the Environment}

The Crampedness of the scene increases the number of false positives of HPE. We, therefore, record the same scenario with and without clutter. We consider clutter to be any object that is not a part of the participant's body. However, clutter is quite objective and therefore we will not be able to define it in a universally applicable way. 

\paragraph{Distance to the camera}

Games developed by SilverFit have a calibration step where the participant is asked to stand at a certain distance from the camera. We, therefore, record the scenario at that specific distance. This ensures that noise introduced by the depth sensor has little effect on the results. The participant is positioned 2 meters away from the camera. \textit{UNSURE, ask someone at SilverFit}.