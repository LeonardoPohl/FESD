\section{Stream pre-processing}

%Firstly, we preprocess the seperate streams of the various depth sensors. This is done by aligning the point clouds and the RGB streams. There are multiple ways to align the point clouds, but we found that, due to the high noise level of the depth data, the best way to align the point clouds is to do it manually. Since we scale the point cloud into meters according to the meter-per-unit ratio provided by the depth camera, manual alignment is a matter of measuring the real-world distances along the axis and then translating the point cloud accordingly. Rotation on the other hand is a lot harder to do manually, so we use the NDT algorithm to align the point clouds, with the measured distances as an initial guess. 

%The NDT algorithm, Normal Distribution Transform algorithm, is a non-linear optimization algorithm, which finds the best rotation and translation to align the point clouds\cite{NDT}. The more data we can provide to the NDT algorithm, the better the alignment will be. Therefore, if we provide the algorithm with the measured distances along the axis, it will be able to find the best rotation without changing the translation.

%The fitness score is provided after the alignment and is a measure of how well the point clouds are aligned. The fitness score is calculated by comparing the aligned point clouds to the original point clouds. The fitness score represents the mean squared distance from each point in the aligned point cloud to its closest point in the static point cloud. Hence, the lower the fitness score, the better the alignment. Depending on the overlap of the point clouds the fitness score can vary a lot. If the point clouds are not overlapping significantly, the fitness score will be higher since the closest point is still far away and will be a bad indicator of a good alignment. Therefore, to validate the alignment of the point clouds we also use the overlap score. The overlap score is calculated by comparing the aligned point clouds to the original point clouds. The overlap score represents the percentage of points in the aligned point cloud that are within a certain distance of a point in the static point cloud. With both metrics, we can validate the alignment of the point clouds.

%The rotation and translation of the point clouds are stored in the file that contains all meta-data about the recording so that we can use them to align the point clouds in the future.