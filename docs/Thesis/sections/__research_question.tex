\section{Research Question}

In this thesis I try to answer if additional RGBD cameras improve human pose detection. For this I try to answer different sub questions, which I will go into in this section.

\subsection{Combination of depth cameras}

For this thesis to yield any result I need some way of synchronising multiple cameras and I need to find out if they work together. In the scope of this thesis I will develop a framework which simply allows multiple cameras to work together seamlessly. I investigate the orbbec astra series (Structured Light), the Intel Realsense L515 (Lidar) and the Kinect for Xbox camera (Structured Light).

These cameras need to be somehow synchronised and the respective coordinate representations lined up.

\subsection{Multi-Cam HPE}

There are multiple ways of combining RGBD camera streams. We will focus on the following:

\paragraph{Combine the skeletons}

HPE is computed for each camera separately. The deduced skeleton is then compared and either an average is formed, or the most accurate is picked.

\paragraph{Combine and compare Skeletons}

Given two cameras that are combined and one that is separate. Using this setup, we form an average of the skeletons of the two combined skeleton and 

\paragraph{Combine depth streams}

Finally if we combine all streams onto one, we might achieve a better result still. This approach has several challenges that need to be faced first. HPE using point clouds is fairly limited, therefore we have to find a way of effectively combining the streams and utilising the data the best way.

It should not be a problem to revert a point cloud into an RGBD image. The challenging part will be choosing the correct angle which results in the best HPE.