\section{Research question}

A major problem with human pose estimation is that it is not possible to tell if the joints of the pose are faulty or not. Using faulty joints can decrease the efficacy of the training effect of the developed games and can make them very frustrating to use and develop. A joint is considered faulty if it is not in the incorrect position, i.e. the distance from the actual position is greater than a chosen threshold, or if it missing from the skeleton.

In this thesis, first, the first research question that I am going to address is; "What problems occur during human pose estimation and how can they be reproduced?". The aim is to find which problems are the most common and which joints are most affected by the errors. Additionally, exercises are designed to emulate different scenarios with ever-increasing difficulty. This will help give an overview of the issues related to human pose estimation and help develop ways to detect these issues. 

Once the issues and error sources that occur during human pose estimation are analysed, the next research question is addressed; "How can I capture a dataset of multiple modalities to analyse if the previous observations about problems during pose estimation are correct?". This allows for the creation of a dataset that can be used to train a model to detect faulty joints.

Using the dataset that is developed I train a model. Using the model I answer the research question posed; "Is it possible to train a model using multi-modal data to determine if a joint is faulty?"