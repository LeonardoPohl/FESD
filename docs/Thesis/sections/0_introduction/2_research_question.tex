\section{Research question}

As mentioned earlier, a major problem with human pose estimation is that it is not possible to tell if the joints are faulty or not. This is a problem for SilverFit, as they want to be able to tell if the joints are faulty or not. Using faulty joints can decrease the efficacy of the training effect of the developed games and can make them very frustrating to use and develop. A joint is considered faulty if it is not in the incorrect position, i.e. the distance from the theoretical position is greater than a chosen threshold, or if it missing from the skeleton.

In this thesis, we first ask what problems occur during human pose estimation and what common error sources are. We aim to find which problems are the most common and which joints are most affected by the errors. This will help give an overview of the issues related to human pose estimation and help develop ways to detect these issues. 

Once we know the issues that occur during human pose estimation, we aim to develop a method that can capture the camera stream in a way that allows us to label the data according to the exercise and environment it was captured in. This will allow us to create a dataset that can be used for future purposes.

Furthermore, we try to find if it is possible, given a joint, the RGB data, and the depth data, to determine if the joint is faulty or not using machine learning. 

Finally, based on the result of the model we attempt to fix the faulty joints in the pose estimation to create a more robust human pose estimation model.
