\chapter{Model development}
\label{sec:model_development}

While there could be multiple approaches to fault estimation, we have chosen to use a deep learning approach. The reason for this is that deep learning has shown to be very successful in many different fields, such as image classification, object detection, and image segmentation. The reason for this is that deep learning can learn the features of the data by itself, without the need for manual feature extraction. This is especially useful in our case, as we have a large amount of data, but we do not know which features are important for the fault estimation.

Other possible solutions could be to use rule-based systems, which use inverse kinematics, or use frame-to-frame joint comparison to detect discreptancies, however, these are quite limited and might result in either too many false positives or false negatives. Furthermore, these rules, such as the frame-to-frame joint comparison, are not always applicable to all types of movements, and therefore might not be able to detect all types of faults in all cases.

\section{Model training}
\label{sec:model_training}

Using this enlarged dataset, we can train a Neural Network to recognise faults in the data. We use the depth data as input, the skeleton data as input, and a combination of both as input. We also experiment with different network layouts, such as a fully connected network, a convolutional network, and a combination of both. We use the augmented data to train the model and the manually checked ground truth to validate the model. We use the validation data to determine the best model and the best network layout. We use the best model to predict the faults in the data. 
\section{Model evaluation}
\label{sec:model_evaluation}

Finally, we evaluate our model by calculating different error metrics such as the mean absolute error, the mean squared error, and the root mean squared error. We also calculate the accuracy of the model, which is the percentage of correctly predicted faults. We also calculate the precision and recall of the model. The precision is the percentage of correctly predicted faults out of all predicted faults. The recall is the percentage of correctly predicted faults out of all faults in the data. We also calculate the F1 score, which is the harmonic mean of the precision and recall. The F1 score is a good indicator of the overall performance of the model.