\section{Previous Work}

In this section I will collect various Papers on the topic of skeleton recognition. Both Rule Based and Machine Learning.

\subsection{Capture with Inter-Part Correlations}

In  Monocular Real-Time Full Body Capture With Inter-Part Correlations \cite{InterPart}:

\begin{itemize}
    \item Training using a simplified dataset, which is focused on Hands, Feet and face separately.
    \item Multi-Dataset training
\end{itemize}

\subsection{DriPE}

In DriPE A Dataset for Human Pose Estimation in Real-World Driving Settings \cite{Dripe}:

\begin{itemize}
    \item Introduce a Dataset with odd angles, changing lighting,
    \item Focusing on Single-person pose Estimation
    \item Introduces a key point based metric for HPE
\end{itemize}

\subsection{BlenSor}

In 2013 in the scope of a PhD Thesis research was conducted\cite{BlenSor-2}, which focuses on the automatic generation of a training and test dataset for autonomous trains in Austria. In their research M. Gschwandtner use the Blender Sensor Simulation Toolbox, BlenSor \cite{BlenSor-1}, to simulate real life sensors, which would be used in the train to simulate certain scenarios, which are too numerous, dangerous or complicated to simulate in real life.