\section{Problem Sets}
\label{sec:problem_set}

Different problem sets are used to create different versions of the model. The problem sets are defined by the number of objects that are considered and are defined as erroneous. There are four different problem sets. The first problem set is the \textit{Joint} problem set. In this problem set, each joint is considered as a single object. The second problem set,  the \textit{Body Part} problem set, is to consider each body part, i.e. the individual arms, legs, torso, and head. The third problem set is the \textit{Half Body} problem set. In this problem set the upper and the lower body are the areas that are considered. Finally, only the whole body is considered in the \textit{Full Body} problem set.

To determine the threshold at which each area is considered faulty, the distribution of joints with errors was calculated for each of the areas was calculated and the 50th percentile was measured. It was found that when considering the full body as an area, the body is considered faulty if more than two joints in the pose are faulty. When considering the lower and upper body, one or more and more than two faulty joints are needed for the area to be considered faulty. For the body parts to be considered faulty one joint within the specific part needs to be faulty, except for the right and left leg, where two joints need to be faulty.

The different problem sets are visualised in Figure \ref{fig:ps}.

\begin{figure}[ht]
  \centering
  \includegraphics[width=\textwidth]{figures/HPE/problem_sets.png}
  \caption[Visualisation of the Problemsets]{The visualisation of the different problem sets, (a) Joint Problem Set, (b) Body Part Problem Set, (c) Half Body Problem Set, and (d) Full Bod Problem Set.}
  \label{fig:ps}
\end{figure}
