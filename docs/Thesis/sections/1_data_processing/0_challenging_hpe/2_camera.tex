\subsection{Camera}

In this section, $REPLACE_WE$ discuss the difficulties that can occur due to the camera. SilverFit uses a predefined camera setup, which is the same for every customer. This setup is tried and tested and has been used for many years. However, the camera setup can still cause difficulties during human pose estimation.

The two main difficulties that can occur with the camera setup are the camera position and the camera angle. Additionally, the camera itself can make the pose estimation process more difficult.

\subsubsection{Distance}

The distance of the camera to the user effects the quality of the pose estimation in multiple ways. Firstly, if the user is too close to the camera, the camera might not get the complete body into frame. The legs and arms might be off the frame preventing them to be estimated properly. Additionally, depth cameras can only detect the depth for a specific range, so if a user is too close they might not be visible to the depth camera.

However, if the user is too far away from the camera, the person might be too small to be detected reliably. There are methods which can detect a human pose from a far distance, however, the further away people get from a camera the more challenging it is. Additionally, as mentioned before, depth cameras operate at different depth ranges. If a user is too far away from the camera it wont be visible to the depth camera. This range varies from camera to camera and also depends on the method of detection.

\subsubsection{Angle}

When considering angles the main focus lays on pitch and roll. For the experiments the yaw is assumed as fixed and directed facing the user, such that the user is in the center. Most human pose estimators are trained without roll in mind. The cameras are usually setup so that any roll is minimised.

Additionally, the pitch may introduce or reduce the occlusion of joints by other joints. It also influences which area is best for human pose estimation. The pitch of a camera depends on what the main application is. For example, if the legs are not considered then the pitch could be used to focus more on the upper body.

\subsubsection{Resolution}

The resolution of the camera, or rather the resolution of the image captured by the camera influences the information that can be gathered about the human and therefore influence the performance of the pose estimation. Also here the application and specifically range are important for choosing the right resolution. If a user is far away a higher resolution is needed to detect the pose reliably.

This is also the case for RGBD cameras. Most RGBD cameras have a set range at which they operate. Additionally, the further away from a depth camera you get, the more noisy the depth stream becomes.