\section{Data Augmentation}


%Once the data is cleaned and we filter recordings with too many missing joints, we can augment the data to simulate a larger amount of data with the ability to create faulty scenarios controlled. One major fault is the seemingly random detachment of the joint to a side. This especially affects the legs and arms, therefore we will have a bias toward limbs with this augmentation. Furthermore, we randomly move the joints with low confidence. Another fault is the disappearance of joints. We use the same bias as with the random detachment, i.e. we take the limb bias, as well as the confidence into consideration.

This phase allows us to create a large amount of data with a controlled amount of faults. This is important since we want to be able to train a model that can detect faults in the data. The augmented data is stored in a separate file so that we can use it to train the model and compare it to the original manually checked ground truth.