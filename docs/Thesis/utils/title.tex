
\title{Improving the Robustness of Human Pose Estimation using Fault Estimation on Multi-Modal Data}
\author{Leonardo Benedikt Pohl}

\date{\today}

This page will be replaced with the actual title page.

\clearpage

\begin{abstract}
    
  From early punch cards to the more recent voice activation, with each new technology, the interaction between humans and computers has become more natural and unobtrusive. One of these newer advances is the interaction with computers based on visual input. Thanks to faster and more available hardware, we can analyse video streams in real time and use the information to enable the interaction between humans and computers. However, this interaction is not always as smooth as we would like it to be. Especially, if humans are in positions that are unnatural pose estimation is not perfect and can lead to errors. This thesis is written in collaboration with SilverFit, a company that develops video games for rehabilitation purposes. SilverFit deals with unnatural poses due to injury or old age in a lot of cases. In this thesis, we collect different scenarios in which human pose estimation can fail. We then develop a method to record different exercises to compile a dataset with both clean and faulty data.  Additionally, the data is augmented based on the confidence values of the joint of the pose. We then use the augmented dataset to train a model that can detect faulty joints in the pose with a confidence rating. Finally, we find approaches to improve the robustness of human pose estimation during streaming. Opening ways to improve the quality of the interaction between humans and computers by improving the robustness of human pose estimation.
   
\end{abstract}
