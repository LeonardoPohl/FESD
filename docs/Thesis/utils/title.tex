
\title{Improving the Robustness of Human Pose Estimation using Fault Estimation on Multi-Modal Data}
\author{Leonardo Benedikt Pohl}
% TITLE
Making the dataset and evaluating the model for fault estimation in human pose estimation

\date{\today}

% \setlength{\voffset}{0cm}
% \setlength{\hoffset}{0cm}

% \includepdf[pages=-]{title.pdf}

% \setlength{\voffset}{-2.54cm}
% \setlength{\hoffset}{-2.54cm}

\clearpage

\begin{abstract}
    
  % First paragraph is not about skeleton
  From early punch cards to the more recent voice activation, with each new technology, the interaction between humans and computers has become more natural and unobtrusive. One of these newer advances is the interaction with computers based on visual input. Thanks to faster and more available hardware, we can analyse video streams in real time and use the information to enable the interaction between humans and computers. However, this interaction is not always as accurate as desired.

  In the scope of this thesis, a method is developed that is capable of capturing and labelling multi-modal data, \textbf{F}ault \textbf{E}stimator for \textbf{S}keleton \textbf{D}etection \textbf{Data} processor (FESDData). Using the dataset that is recorded and labelled using FESDData, I developed a model which aims to detect if an error occurs, \textbf{F}ault \textbf{E}stimator for \textbf{S}keleton \textbf{D}etection \textbf{Model} (FESDModelv2). FESDModelv2 is an improvement over an older model that was also developed in the scope of this thesis. The new version uses transfer learning to extract the features from the data. This thesis is written in collaboration with SilverFit, a serious gaming company that develops video games for rehabilitation purposes. SilverFit deals with unnatural poses due to injury or old age in a lot of cases. 
% In collaboration with SilverFit not in abstract put it in more general
% name general applications
% Dataset should have a name
% some prelimenary results
  The result of my research shows that the collected and labelled data is not enough to create a generalised model for error detection. There are too many parameters  However, the developed method for data collection might still prove useful in future applications. This opens ways to improve the quality of the interaction between humans and computers by improving the robustness of human pose estimation.
   
\end{abstract}
