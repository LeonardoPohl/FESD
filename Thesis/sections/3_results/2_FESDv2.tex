\section{FESDModelv2 Results}

Similar to FESDModelv1, FESDModelv2 was first trained on 64x64 pixels input images and subsequently on 200x200 pixels images.

\subsection{Results for low-resolution images}

The results of the testing after 50 epochs of training for FESDModelv2 on low-resolution images as listed in Table \ref{tab:res_v2}.

The results of the Body Part and Joint problem set have been omitted since the low resolution of the input images prohibited reasonable training for the datasets.

    \begin{table}[!htbp]
        \caption[Test Results of FESDModelv1]{The test results of FESDModelv1 after 50 epochs of training on the Full Body, Half Body and Body Parts dataset. Showing the Percentage of Positive Guesses (PPG), the Accuracy (Acc), the F1 Score, and the Cohen's Kappa Coefficient ($\kappa$).}
        \label{tab:res_v1}
        \centering
        \begin{tabular}{p{0.14\linewidth}p{0.12\linewidth}p{0.12\linewidth}p{0.12\linewidth}p{0.12\linewidth}}
\hline
{} &   PPG &  Acc &   F1 &    $\kappa$ \\
Problem Set   &       &          &      &           \\
\hline 
Full Body   & $0.42$ &	$0.78$ &	$0.68$ &	$0.52$ \\
Half Body   & $0.42$ &	$0.81$ &	$0.77$ &	$0.61$ \\
Body Parts  & $0.17$ &	$0.78$ &	$0.27$ &	$0.14$ \\

\hline
\end{tabular}

    \end{table}


The best results are achieved on the Half Body problem set with an F1 Score of $0.50$. Similar to FESDModelv1, FESDModelv2 achieves the best results when predicting the error label for the lower half of the body. This is also reflected in the area under the ROC curve. The AUC of the lower body prediction is $0.49$ while the AUC of the upper body is $0.42$.

\begin{figure}[htbp]
  \centering
  \begin{subfigure}[b]{0.4\linewidth}
      \centering
      \includegraphics[width=\textwidth]{figures/Results_lo/v2/confusion/full_together.png}
      \caption[]{Full Body Problem Set}
      \label{fig:fb_conf}
  \end{subfigure}
  \hfill
  \begin{subfigure}[b]{0.4\linewidth}
      \centering
      \includegraphics[width=\textwidth]{figures/Results_lo/v2/confusion/half_together.png}
      \caption{Half Body Problem Set}
      \label{fig:hb_conf}
  \end{subfigure}
  \caption[Confusion Matrices of FESDModelv2 (64x64 pixels input resolution)]{The confusion Matrices of FESDModelv2 for the Full Body and Half Body problem sets trained on input images of 64x64 pixels resolution.}
  \label{fig:conf_v2}
\end{figure}

\FloatBarrier

%
%
%
%
%
%
%
%
%
%
%
%
%
%
%

\subsection{Results for higher resolution images}

The results of the testing after 20 epochs of training for FESDModelv2 on 200x200 pixels resolution images of the Full Body, Half Body, Body Parts and Joints problem sets, respectively, as listed in Table \ref{tab:hi_res_v2}.

    \begin{table}[!htbp]
        \caption[Test Results of FESDModelv1]{The test results of FESDModelv1 after 50 epochs of training on the Full Body, Half Body and Body Parts dataset. Showing the Percentage of Positive Guesses (PPG), the Accuracy (Acc), the F1 Score, and the Cohen's Kappa Coefficient ($\kappa$).}
        \label{tab:res_v1}
        \centering
        \begin{tabular}{p{0.14\linewidth}p{0.12\linewidth}p{0.12\linewidth}p{0.12\linewidth}p{0.12\linewidth}}
\hline
{} &   PPG &  Acc &   F1 &    $\kappa$ \\
Problem Set   &       &          &      &           \\
\hline 
Full Body   & $0.42$ &	$0.78$ &	$0.68$ &	$0.52$ \\
Half Body   & $0.42$ &	$0.81$ &	$0.77$ &	$0.61$ \\
Body Parts  & $0.17$ &	$0.78$ &	$0.27$ &	$0.14$ \\

\hline
\end{tabular}

    \end{table}


FESDModelv2 achieves the best F1-Score of all models when trained on images with a higher resolution. The best results are achieved on the Half Body problem set with an F1-Score of 0.71.

\begin{figure}[htbp]
  \centering
  \begin{subfigure}[b]{0.35\linewidth}
      \centering
      \includegraphics[width=\textwidth]{figures/results_hi/v2/confusion/full_together.png}
      \caption[]{Full Body Problem Set}
      \label{fig:hi_fb_conf}
  \end{subfigure}
  \hfill
  \begin{subfigure}[b]{0.35\linewidth}
      \centering
      \includegraphics[width=\textwidth]{figures/results_hi/v2/confusion/half_together.png}
      \caption{Half Body Problem Set}
      \label{fig:hi_hb_conf}
  \end{subfigure}
  \hfill
  \begin{subfigure}[b]{0.35\linewidth}
      \centering
      \includegraphics[width=\textwidth]{figures/results_hi/v2/confusion/body_parts_together.png}
      \caption{Body Part Problem Set}
      \label{fig:hi_bp_conf}
  \end{subfigure}
  \hfill
  \begin{subfigure}[b]{0.35\linewidth}
      \centering
      \includegraphics[width=\textwidth]{figures/results_hi/v2/confusion/joints_together.png}
      \caption{Joint Problem Set}
      \label{fig:hi_jt_conf}
  \end{subfigure}
  \caption[Confusion Matrices of FESDModelv2 (200x200 pixels input resolution)]{The confusion Matrices of FESDModelv2 for the different problem sets trained on input images of 200x200 pixels resolution.}
  \label{fig:hi_conf_v2}
\end{figure}

\FloatBarrier
