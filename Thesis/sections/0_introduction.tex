\chapter{Introduction}
\pagenumbering{arabic}
\setcounter{page}{1}
% general introduction to the problem
Human pose estimation (HPE), or skeleton detection, aims at detecting the pose or skeleton of a person based on visual information and/or depth information only. It finds many applications, from games to medical applications\cite{kumarapu2020animepose, ClinicalApplicationChen, MedicalAnimation}. 

\textit{Gaming and entertainment} are one of the most common applications of HPE. Games can use HPE in a way that makes the interaction between humans and computers very natural. One of the systems that kickstarted the use of depth cameras in games was the Xbox Kinect by Microsoft, which uses a depth camera, the Microsoft Kinect, to track the movements of the player and used this information to control the games.

HPE also finds application in \textit{Autonomous driving}, which has been in development ever since humans replaced horses with cars\cite{OldAutoDrive}. However, the development of autonomous driving has been challenging. The main reason for this is that autonomous driving requires a lot of information about the environment, e.g. the road, pedestrians, other vehicles, etc. In some cases, cars need to be able to estimate the pose of a human to make a decision. The posture of a human can be used to determine the action and therefore the future trajectory of the person. 

To exactly emulate human movements in \textit{Animations}, animators can either manually move the joints of a digital skeleton or they can use HPE of real human actors. The manual creation of realistic movement is oftentimes very time-consuming and also error-prone. Therefore, animators often use HPE of real human actors leading to digital skeletons and movements that often are more accurate.

HPE is also used in medical applications, to aid in the \textit{Rehabilitation} of patients as well as to improve the quality of life of elderly people. Here HPE can be used to detect the pose of a person and provide feedback on the pose of the person during exercises\cite{ClinicalApplicationChen}. 

Additionally, HPE can be used as a \textit{Diagnostic tool} for all ages. For example, to detect Cerebral palsy (CP) usually, an MRI is used using human observations. However, MRIs are expensive and the expert knowledge to detect CP requires long training. HPE offers a low-cost alternative for the detection of CP risk using automatic movement assessment with comparable performance to standardised CP risk measurers\cite{Stenum2021ApplicationsOP}.

Throughout these applications, HPE is a critical component that is required to be accurate. However, HPE is a difficult task that is prone to errors\cite{HPEIsHard}, which can be caused by different factors, such as the environment, the camera, and the person. These errors can cause the joints of the estimated pose to be in incorrect positions or missing, thereby hampering the human-computer interaction. The goal of this thesis is to develop a method that allows for the detection of these errors such that, the pose information can be improved by post-processing.

One of the main challenges for HPE is the diversity of users and posture appearances. If the posture is very different from what the pose estimation model expects then errors might occur. This is especially true for applications that are designed for rehabilitation and exercise purposes. In these applications, the user is often elderly and has limited mobility. This is precisely the case for games developed by SilverFit\footnote{\url{https://www.silverfit.com/en/}}. SilverFit is a serious gaming company that develops games for rehabilitation with a special focus on geriatric patients. In their games, SilverFit uses HPE to detect the pose of the player and use it to control the game to make exercising more enjoyable while promoting activity. The research conducted in this thesis is inspired by the challenges that the applications from SilverFit have to face and aims at improving HPE by giving feedback on the detection of errors occurring during HPE in their games.

For some exercises designed by SilverFit and other companies, HPE is not sufficiently reliable to create an enjoyable experience for patients in every environment. Especially sedentary exercises often cause the HPE to fail or to produce unreliable results.

To analyse and mitigate the errors that occur during HPE three research questions are formulated. 


\begin{enumerate}[label=(\Alph*)]
  \item What typical errors occur during HPE and how can these errors be reproduced in controlled scenarios?
  \item How can a multimodal HPE dataset be captured and labelled?
  \item Can we train a deep neural network (DNN) to detect errors in human pose estimations using RGB-D and pose input data? 
  \end{enumerate}

To answer the research questions, first, the most common HPE error sources are analysed. During this analysis, different factors are discussed which may influence the performance of human pose estimation. Based on these factors exercises are designed to emulate different scenarios with different levels of challenges for HPE. This is discussed in Section \ref{sec:errors}.

In Section \ref{sec:data_acquisition} a custom tool, FESDData, was designed and implementeted to capture and label multi-modal data, consisting of RGB, Depth and Pose data, and metadata such as exercise name and frame time stamps. The tool allows capturing predefined exercises and labelling each joint of each captured pose data frame with error labels. The collected data forms the dataset, FESDDataset.

In Section \ref{sec:model_development}, two DNN models for HPE error detection are proposed, FESDModelv1 and FESDModelv2. The FESDDataset is used as the training set for these models using a data loader that performs the necessary preprocessing steps, such as data augmentation, data balancing and resizing of the modalities. FESDModelv1 and FESDModelv2, are trained to detect HPE errors at different levels of difficulty. The experimental setup is described in Section \ref{sec:model_training}. Finally, the experimental results are discussed in Section \ref{sec:results}.

% In this thesis, there are a few ambiguities due to legacy naming and the influence from different sources. If the text refers to \textit{Skeleton Detection} it refers to \textit{Human Pose Estimation using a Kinematic representation}, i.e. using joints and bones. Unless stated otherwise, Human Pose Estimation refers to this representation. Furthermore, \textit{Fault Estimation} or \textit{Fault Detection} refers to the detection of errors or anonamlious position of joints as discussed in Section \ref{sec:data_labeling}.