\chapter[FESDModel]{FESDModel - Preliminary Experiments for Model development}
\label{sec:model_development}

While there could be multiple approaches to fault estimation, a deep-learning approach using deep convolutional neural networks(CNN) has been chosen. The reason for this is that CNNs have shown to be very successful in many different fields, especially in computer vision tasks, such as image classification, object detection, and image segmentation.

We speculate that other possible solutions could be to use rule-based systems, which use inverse kinematics, or use frame-to-frame joint comparison to detect discrepancies, however, we expect that these are quite limited and might result in either too many false positives or false negatives, but these are speculations and need to be further investigated to form a clear understanding of alternative approaches for error detection. As no research has been found to actual error detection this is left as future work as such an investigation is a task in its own right. 

The closest approach to error detection was proposed by Joao Carreira et. al for their Iterative Error Detection algorithm. However, in their work, the authors do not detect errors but rather determine the best change that can be applied to an estimated pose to achieve a better pose. Therefore, this approach cannot efficiently be used to determine an error in a scene, as a correction does not necessarily mean that there is an error.

Here two different models are proposed as they could in general be designed for this kind of task on a dataset consisting of three different modalities, RGB, Depth, and Pose Data.

\section{Model architecture}
\label{sec:model_architecture}

For solving the problem of error estimation, two different model architectures are proposed. The first model architecture, FESDModelv1, is a deep convolutional neural network that uses the RGB, Depth, and Pose data as separate inputs. This model can be seen in Figure \ref{fig:model_architecture_v1}.

\begin{figure}[htbp]
  \centering
  \includegraphics[width=.8\linewidth]{figures/Model/FESD.png}
  \caption[FESDModel architecture version 1]{FESDModelv1 architecture with three different inputs; RGB, Depth and Joint data. With 'S' as the image size. After three convolutions the three streams are concatenated to be passed into three fully connected layers with ReLU activation functions. In this example network, the model calculates the Joint problem set, therefore the output is a 1D 40 tensor consisting of pairs of boolean values. Each of the 20 pairs corresponds to an individual joint.}
  \label{fig:model_architecture_v1}
\end{figure}

The second model architecture, FESDModelv2, utilises transfer learning to extract the features of the input data using a pre-trained model. The architecture of FESDModelv2 can be seen in Figure \ref{fig:model_architecture_v2}. Both models are trained to predict the error labels for each joint. The error labels are the same as the error labels used in the data labelling explained in Section \ref{sec:data_labeling}. The fully connected layers of both networks use intermittent rectified linear unit (ReLU) activation functions to combat the vanishing gradient problem by passing only the values which are greater than zero into the next layer.

\begin{figure}[htbp]
  \centering
  \includegraphics[width=.5\linewidth]{figures/Model/FESDv2.png}
  \caption[FESDModel architecture version 2]{FESDModelv2 architecture with transfer learning. The input is merged into a single RGB image and passed into a feature extractor. With 'S' as the image size. The feature extractor is a pre-trained EfficientNet v2 S. The output of the feature extractor is passed into two fully connected layers with ReLU activation functions. In this example network, the model calculates the Joint problem set, therefore the output is a 1D 40 tensor consisting of pairs of boolean values. Each of the 20 pairs corresponds to an individual joint.}
  \label{fig:model_architecture_v2}
\end{figure}

While FESDModelv1 uses the data as it is stored in the dataset, FESDModelv2 merges the data into a single RGB image. This is done using the feature extractor, which is originally trained on RGB images. The data is merged by assigning each modality to a seperate channel of an RGB image. The RGB image is transformed into greyscale and assigned to the red channel of the RGB input image, the depth image is scaled to a value between 0 and 255 and assigned to the green channel, and the joint coordinates are depicted as white squares on an image, aligned to both the RGB and Depth image, assigned to the blue channel.

In total eight models were developed and trained. Four models were trained using FESDModelv1 and four models were trained using FESDModelv2. Each of the models corresponds to the problem sets A-D, as introduced in Section \ref{sec:problem_set}. And the model for each respective problem set is trained using both FESDModelv1 and one model for each problem set is trained using FESDModelv2.

Consequently, the output of the models varies depending on the problem set. Based on the problem sets, their respective problem areas and the error labels as discussed in Section \ref{sec:data_labeling}, the Full Body, Half Body, Body Part, and Joint problem sets have an output vector of size 2, 4, 12, and 40 respectively. While more detailed information exists about the error for the Joint problem set, it was decided that the simplified "Error"/"No Error" label is being predicted.

FESDModelv2 uses a neural network which was pretrained on ImageNet as a feature extractor. Multiple candidate networks have been compared, which can be seen in Figure \ref{fig:network_comparison}. One of the target applications of the model is to be used in a real-time application so that error handling can be conducted. Consequently, a lightweight model which does not impact the performance much is preferred. Therefore, the models are compared by the number of floating-point operations (FLOPS) to their Accuracy on ImageNet-1K. Table \ref{tab:network_comparison} shows the top 5 models according to their accuracy and performance. 

\begin{figure}[htbp]
  \centering
  \includegraphics[width=.8\linewidth]{figures/network/networks.png}
  \caption[Network comparison]{The comparison of different networks by their GFLOPS and their Top-5 Accuracy. The models are sorted by their GFLOPS and their Top-5 Accuracy(Source: \url{https://pytorch.org/vision/main/models.html} on 08/05/2023). The models are EfficientNet V2 S, ConvNeXt Base, EfficientNet B6, Swin V2 B, and EfficientNet V2 M. Additionally, AdamNet, ResNet-50 and Inception-v3 are added as a reference.}
  \label{fig:network_comparison}
\end{figure}

\begin{table}[htbp]
  \caption[Top 5 models for Accuracy and Performance]{The top 5 models according to their accuracy and performance. The models are sorted by their GFLOPS and their Top-5 Accuracy(Source: \url{https://pytorch.org/vision/main/models.html} on 08/05/2023). The models are EfficientNet V2 S, ConvNeXt Base, EfficientNet B6, Swin V2 B, and EfficientNet V2 M.}
  \label{tab:network_comparison}
  \centering
  \begin{tabular}{lrrrr}
    \hline
            Weight &  Acc@1 &  Acc@5 &   Params &  GFLOPs \\
    \hline
  EfficientNet V2 S & 84.228 & 96.878 & $2.15 \times 10^7$ &   8.370 \\
      ConvNeXt Base & 84.062 & 96.870 & $8.86 \times 10^7$ &  15.360 \\
    EfficientNet B6 & 84.008 & 96.916 & $4.30 \times 10^7$ &  19.070 \\
          Swin V2 B & 84.112 & 96.864 & $8.79 \times 10^7$ &  20.320 \\
  EfficientNet V2 M & 85.112 & 97.156 & $5.41 \times 10^7$ &  24.580 \\
  \hline
  \end{tabular}
\end{table}

EfficientNet v2 was chosen since it proved to be the most performant while being the most accurate of the networks that were analysed. In particular, the small variant with $2.15 \times 10^7$ Parameters and a Top-1 Accuracy of $84.228\%$\cite{tan2021efficientnetv2}.  While EfficientNet V2 M out-performs EfficientNet V2 S in terms of Top-1 Accuracy, the number of additional parameters needed to achieve a better accuracy outway the performance bonus that EfficientNetv2 S brings with it. EfficientNetv2 is a convolutional neural network, which optimises training speed and parameter efficiency and improves upon EfficientNet\cite{tan2020efficientnet}. The main focus of EfficientNet is the scaling of the model in width, depth and resolution of the input image. 

\FloatBarrier
\section{Data preparation}
\label{sec:data_preparation}

To successfully train FESD three steps are taken before training can begin, data augmentation, data merging, and data balancing. The data augmentation is done to ensure that the model is robust to different variations in the data. The data merging is done to combine the different modalities into a single tensor. The data balancing is done to ensure that the model is not biased toward any particular error label.

The finished data preparation pipeline can be seen in Figure \ref{fig:data_preparation_pipeline}.

\begin{figure}
  \centering
  \includegraphics[width=\linewidth]{figures/Model/Data_Preparation_Pipeline.png}
  \caption[Data preparation pipeline]{The data preparation pipeline. The data is first augmented, then merged, and finally balanced.}
  \label{fig:data_preparation_pipeline}
\end{figure}

The images that are stored by \textit{FESDData} are inherently the same size. The joints, however, are stored within a JSON file containing the coordinates of each joint in 2D and 3D. To further process the 2D joint data is drawn on an image that
 has the same dimensions as the RGB and Depth image.

\subsection{Data augmentation}

Four different augmentations are applied to the data to generalise the data. The first augmentation is flipping the data. The RGB image, the depth image, and the joint image are flipped horizontally. Furthermore, the ground truth data is flipped, as labels refer to the left or right side of the body, which would no longer coincide with the data that is passed into the network.

Additionally, the images are cropped at random while keeping the positions of the joints and a margin around the joints visible. This ensures that the model is robust to different positions of the user in the image. 

Finally, at random Gaussian noise is applied to the RGB image and the depth image. This further improves the robustness of irregular data.

The augmentations can be seen in Figure \ref{fig:data_preparation_pipeline} where they are applied to a sample frame from the dataset.

\subsection{Data merging}

While there are different ways of applying transfer learning to the problem at hand, we chose to merge the different modalities into a single tensor. This allows us to use the EfficientNet v2 as a feature extractor for all modalities, without the added computations of extracting the features of each modality individually. The data is merged by assigning each modality to a channel in an RGB image.

In Figure \ref{fig:data_preparation_pipeline} we can see the different modalities as they are merged into a single tensor. The RGB image is transformed into greyscale and assigned to the red channel, the depth image is scaled to a value between 0 and 255 and assigned to the green channel, and the joint coordinates are assigned to the blue channel. The data is then passed into the EfficientNet v2 as a single RGB image.

\subsection{Data balancing}

In the ordinary case, human pose estimation is not meant to produce faulty results. In the selected exercises it is aimed to produce faulty results. However, this does still not produce a balanced dataset. In section \ref{sec:dataset}, the statistics of the dataset are shown. Especially, in Figure \ref{fig:err_dist_joint_pie, fig:err_dist_limb_pie, fig:err_dist_half_pie, fig:err_dist_full_pie} it can be seen that the dataset is imbalanced. Most notably for the problem set \textit{Half} and \textit{Full} where the error label \textit{No Error} is overrepresented.

To balance the dataset, frames are sampled using a Weighted Random Sampler for each batch of the training. The weights for the samples are calculated based on the occurrence of the error labels in the dataset. While only considering the whole body as a single object, the calculation of the weights is simple. For each frame, the error label is counted and the inverse of the count is used as the weight for the frame. This ensures that the model is not biased toward any particular error label by oversampling the frames which contain an error.

However, for the other problem sets the calculation of the weights is more complex. In the other problem sets each frame contains an error for each area, e.g. when considering the Half-Body problem, the upper and lower body 2 errors. To successfully balance the dataset for each area four weights would need to be created and balanced, i.e. the upper and lower body have an error the upper body is faulty and the lower body is not, etc. This would oversample some frames while undersampling others. In the other problem sets this is far more visible. Therefore, it was decided to consider the sum of errors is considered as a single error label. This means that frames that have the same number of erroneous areas are weighed the same.

An example of the distribution of errors before and after balancing can be seen in Figure \ref{fig:balance} for each problem set.

% In previous sections, we have explained the structure of the data and how the data is labeled. In this section, we will explain how the data is prepared for training. The data preparation consists of two parts, data augmentation, and data splitting.

% After the data has been split and augmented, the data is stored in tensors of a fixed size. If an image is too small we apply a bilinear interpolation to resize the image. If an image is too large we crop the side. The images are resized to $300 \times 300$ pixels.

% \subsection{Data augmentation}
% \label{sec:data_augmentation}

% To ensure that the model is robust to different variations in the data, the data is augmented. We apply three different forms of augmentation to the data. The first form of augmentation is flipping the data. The RGB image, the depth image, and the relative joint coordinates are flipped horizontally. Furthermore, we apply random and non-random cropping to the visual data. We ensure that none of the joints are outside the image. We also apply random padding when the image is cropped. The final augmentation is Gaussian noise. We apply Gaussian noise to the RGB image and the depth image. In Figure \ref{fig:augmentation} we can see the different augmentations.

% \begin{figure}
%     \includegraphics[width=.33\textwidth]{figures/Model/Augmentation/Original.png}\hfill
%     \includegraphics[width=.33\textwidth]{figures/Model/Augmentation/Flipped.png}\hfill
%     \includegraphics[width=.33\textwidth]{figures/Model/Augmentation/GaussianBlur.png}
%     \\[\smallskipamount]
%     \includegraphics[width=.33\textwidth]{figures/Model/Augmentation/Cropped.png}\hfill
%     \includegraphics[width=.33\textwidth]{figures/Model/Augmentation/CroppedPad50.png}\hfill
%     \includegraphics[width=.33\textwidth]{figures/Model/Augmentation/RandomCropped.png}
%     \caption[Data Augmentation]{Augmentation applied to a sample frame. The augmentations are; Horizontal flipping, Gaussian Blur, Exact Cropping, Exact Cropping with padding, and random Cropping respectively.}\label{fig:augmentation}
% \end{figure}


% \subsection{Data splitting}

% The data is split into two parts. The first part is the training data. The training data is used to train the model. To test the validity of the model itself, we split the data by exercise. We define 4 exercises, which are not used during training. We chose the exercises such that each exercise is in a different difficulty category, i.e. one of the exercises for testing has a trivial difficulty, one is considered as easy, and so on. With this, we try to find if our model performs better or worse with increased difficulty on unseen data.

\section{Model training}
\label{sec:model_training}

To train a neural network with supervised learning, you first pass the input data into the network and forward it through the defined layers. Then a loss is calculated an is is back propagated through the network to adapt the weights accordingly. 

As mentioned earlier, error or anomaly detection for human pose estimation can be seen as a multi-class multi-object classification problem. The loss function needs to be chosen such that it best reflects the data and the efficacy of the model. In most classification problems \textit{Cross Entropy Loss} is calculated and propagated through the network to adapt the weights and convolutions accordingly. Cross entropy loss calculates the soft max of the result and compares it to the ground truth. The soft max calculates the probability of each index in a list based on the value at that index. Cross Entropy Loss penalises the results based on the probability of the target class.

However, since REPLACE_WE have multiple objects, in REPLACE_OUR case the different joints, REPLACE_WE have to slightly change the loss function. If REPLACE_WE were to use the Cross Entropy loss on the results as REPLACE_WE got them now, the model would try to optimise toward a single class for all joints, which does not correctly reflect the desired result. Therefore, REPLACE_WE apply the Cross Entropy loss on every joint and propagate them individually.

\textbf{Show formula of loss function}
