\chapter[FESDData]{FESDData - Data Acquisition and Labelling of FESDDataset}
\label{sec:data_processing}

One of the most important aspects of machine learning is data acquisition since a machine learning algorithm can only be as good as the data provided. In this thesis, a custom tool is developed which is capable of capturing and labelling multi-modal data. The tool is called \textbf{F}ault \textbf{E}stimator for \textbf{S}keleton \textbf{D}etection \textbf{Data} processor (FESDData). It is capable of capturing RGB-D data (RGB and the corresponding depth) from multiple different RGB-D cameras as well as pose data calculated by Nuitrack, which was developed by 3DiVi Inc\footnote{\url{https://nuitrack.com/}}. Nuitrack was chosen as the pose estimation backbone since it provided the support of multiple different camera models as well as the ability to extract the RGB-D data and pose data simultaneously.

In addition to Nuitrack, FESDData is capable of calculating the pose information using OpenPose after the recording is done. This offers future support for other pose estimation backbones. However, OpenPose is not used in the experiments for this thesis.

FESDData is designed to be easily used by anyone and to require minimal setup or tweaking. It allows for the rapid capture of RGB-D datasets with the pre-labelling of multiple different exercises. The parameters can be adapted to capture datasets for many different applications, e.g. action detection, where HPE is used to identify an action. 

For this research, selected exercises which are designed to emulate different scenarios with different levels of difficulty. These selected exercises are discussed in Section \ref{sec:errors}. The labelling of errors in the data requires domain-specific knowledge and is therefore done manually. The resulting dataset, FESDDataset, is discussed in Section \ref{sec:dataset}. It consists of the RGB-D data as well as the pose data and the labels for the errors for each of the joints in the pose.

In this chapter, the approach to the design of the exercises is discussed by showing the challenges of HPE. Furthermore, the data acquisition and labelling process is discussed. Finally, the features of the dataset are presented.

\section{Challenges in Human Pose Estimation}
\label{sec:errors}

In this chapter, possible faults and difficulties that occur during human pose estimation are discussed. These difficulties are caused by different factors, such as the environment, the camera, the person, and the software. 

These difficulties are important to understand, as they can cause the joints to be in the incorrect position or missing. Leading to an incorrect estimation of the pose estimation, and therefore, the human-computer interaction will be hampered.

\section{Environment}

\subsection{Background}

The background of the scene can cause difficulties in the human pose estimation process.

\subsection{Crampdness}

\subsection{Lighting}


\subsection{Camera}

In this section, $REPLACE_WE$ discuss the difficulties that can occur due to the camera. SilverFit uses a predefined camera setup, which is the same for every customer. This setup is tried and tested and has been used for many years. However, the camera setup can still cause difficulties during human pose estimation.

The two main difficulties that can occur with the camera setup are the camera position and the camera angle. Additionally, the camera itself can make the pose estimation process more difficult.

\subsubsection{Distance}

The distance of the camera to the user effects the quality of the pose estimation in multiple ways. Firstly, if the user is too close to the camera, the camera might not get the complete body into frame. The legs and arms might be off the frame preventing them to be estimated properly. Additionally, depth cameras can only detect the depth for a specific range, so if a user is too close they might not be visible to the depth camera.

However, if the user is too far away from the camera, the person might be too small to be detected reliably. There are methods which can detect a human pose from a far distance, however, the further away people get from a camera the more challenging it is. Additionally, as mentioned before, depth cameras operate at different depth ranges. If a user is too far away from the camera it wont be visible to the depth camera. This range varies from camera to camera and also depends on the method of detection.

\subsubsection{Angle}

When considering angles the main focus lays on pitch and roll. For the experiments the yaw is assumed as fixed and directed facing the user, such that the user is in the center. Most human pose estimators are trained without roll in mind. The cameras are usually setup so that any roll is minimised.

Additionally, the pitch may introduce or reduce the occlusion of joints by other joints. It also influences which area is best for human pose estimation. The pitch of a camera depends on what the main application is. For example, if the legs are not considered then the pitch could be used to focus more on the upper body.

\subsubsection{Resolution}

The resolution of the camera, or rather the resolution of the image captured by the camera influences the information that can be gathered about the human and therefore influence the performance of the pose estimation. Also here the application and specifically range are important for choosing the right resolution. If a user is far away a higher resolution is needed to detect the pose reliably.

This is also the case for RGBD cameras. Most RGBD cameras have a set range at which they operate. Additionally, the further away from a depth camera you get, the more noisy the depth stream becomes.
\section{Person}

Finally, one of the main error sources of human pose estimation is the person. The person can cause difficulties in the human pose estimation process by moving, wearing specific clothes, or having a different body posture. Body posture is of special importance for SilverFit since SilverFit specialises in games for rehabilitation and elderly people. Elderly people have different body postures than the average person, which can cause difficulties in the human pose estimation process.

\subsection{Clothes}

As mentioned earlier, most RGBD cameras use infrared light to determine the depth of the scene. This means that the clothes of the user can cause more or less absorption of light and therefore influence the detected depth. This can cause the joints to be detected in the wrong position or not at all. This is especially the case for dark clothes, as they absorb more light than light clothes.

Furthermore, bulky clothes or skirts and dresses may influence the pose, since the exact position of the legs is not visible.

\subsection{Training Equipment}

To make exercises more challenging some physiotherapists use additional training equipment. This could be weights that are held in the hand or weights that are attached to the ankles. These weights change the outline of the body and therefore influence the pose estimation. 

\subsection{Exercises}

Finally, the most important factor is the exercise that is carried out. In this section, we define some exercises that are easy to detect as well as some exercises that are difficult to detect. We also discuss the difficulties that cause the exercises to pose issues for human pose estimation.

These exercises might not be the most realistic, but they represent common issues with pose estimation in a reproducible manner. Furthermore, these exercises are not too difficult to perform, which makes them suitable for testing the pose estimators. The difficulty rating of the exercise might not reflect the difficulty of the exercise for the user, but it does reflect the difficulty of the exercise for the pose estimator.

\subsubsection{Trivial Exercises}

In the most trivial case, the person is standing still with their arms stretched to the side. In this case, the person is not moving and the joints are not changing position. This is the easiest case for human pose estimation, as the joints are always in the same position. However, this is not a realistic case, as the person is not exercising but it offers a baseline for the other exercises.

The arms stretched to the side reduce the possibility of occlusion, as the side of the body is not blocking the joints of the arms. Furthermore, the person is standing still, which reduces the possibility of occlusion as well.

\subsubsection{Easy Exercises}

Easy exercises are essential for creating a good baseline of how the pose estimators should work. These exercises are not too difficult to detect and are therefore good for testing the pose estimators. The exercises include no self-occlusion and are recorded in a standing position, which is generally the easiest position to detect.

\paragraph{Raising the arms to the side}

The first easy exercise is only a small step up from the trivial exercise. In this exercise, the person raises their arms to the side. This exercise is easy to detect, as the arms are raised to the side and the joints are not occluded by the body. Furthermore, the person is standing still, which reduces the possibility of occlusion as well. However, now the arms are moving. This should not pose a problem for the pose estimators, as the arms are not moving too fast. However, it is important to note that the arms are moving, as this can cause issues in some pose estimators.

\paragraph{Raising the arms to the front}

A slightly more challenging exercise is when the user raises the arms to the front. This exercise is slightly more challenging than the previous exercise, as the arms are now occluding themselves.

\subsubsection{Medium Exercises}

Exercises performed in a seated position are harder to detect. Medium exercises focus on exercises, which are performed in a seated position. These exercises only involve arm movement which is easier to detect.

\paragraph{Raising the arms to the side}

The first medium exercise is similar to the easy exercise, but now the person is sitting down. Additionally, the user will be holding weights to increase the difficulty of the exercise.

\paragraph{Raising the arms to the front}

As with the previous exercise, the user will be sitting down and holding weights. However, now the user will be raising the arms to the front. 

\paragraph{Crossing the arms}

The last medium exercise is when the user crosses the arms in front of the body. This exercise is slightly more challenging than the previous exercise, as the arms are now occluding themselves, as well as the upper body.

\subsubsection{Difficult Exercises}

Difficult exercises are exercises that are performed in a standing position and involve leg movement. Leg joints are harder to detect than arm joints and therefore pose a greater challenge for the pose estimators. These exercises will be in a seating position and with a difficult posture, such as leaning forward. The difference in posture aims at creating a realistic representation of real-world exercises.

\paragraph{Raising the knee}

The first exercise is when the user raises the knee. This exercise had to be reworked at SilverFit to function well since the pose estimation was too unreliable. From a neutral sitting position with knees at around 90 degrees, the user lifts the knee to a 45-degree angle. Meanwhile, the arms are down to the side.

\paragraph{Raising the knee leaning forward}

Additionally to raising the knee, the user will now lean forward, to emulate bad posture.

\paragraph{Raising the knee leaning forward facing away from the camera}

The user will now lean forward and the body will face away from the camera at a 20-degree angle. This leads to more occlusion and the complete lack of visibility of one of the arms. 
\section{Data acquisition}
\label{sec:data_acquisition}

Different modalities were captured to create a dataset that reproduces a real-world application of human pose estimation for RGBD cameras. The different modalities are RGB data, depth data, and joint data. While the Nuitrack SDK offers to capture the data from the RGBD cameras and the joint data, the recorded files cannot, at the time of writing, be read without using the Nuitrack SDK. Therefore, FESDData, a custom RGBD+HPE capturing and labelling tool was developed. 

FESDData has two main uses which are interlocked. Firstly, it allows capturing predefined, as well as custom, exercises repeatedly automatically making the capturing experience when capturing many different actions more comfortable. Secondly, it allows reviewing and labelling the captured data with error labels. The lightweight nature of FESDData allows it to seamlessly capture both the RGBD stream and the skeleton data at the same time while ensuring a stable fast framerate. The dataset that is used by FESDModel was captured at a framerate of 30 frames per second.

\section{Data layout}

In this section, we discuss the data layout that is used to store the data. This makes it possible to reuse the data in the future for any other application. As mentioned earlier, the data is made up of multiple different streams or modalities. There are two separate visual streams, the RGB stream, and the depth stream, as well as the estimated human pose, the time stamps, and the recording metadata.

The visual streams are normalised and combined into a single file. We use OpenCV to store the RGB and the depth data into a single matrix and after the stream into a single file per frame. The RGB data is normalised to have values between 0 and 1, whereas the depth data is stored in meters. To improve the size of the data as well as the read and write speed, we store the visual data as binary files instead of using the human readable FileStore system provided by OpenCV.

The human pose estimated by Nuitrack as well as the error labels are stored in a separate json file. The separate frames are stored in a list of frames. Each frame contains a list of all people that were detected. Each person contains a list of joints as well as an error label. In some cases, a person might be detected completely wrong, but we do not clean this data as it is still valuable data for model development. Each joint is stored with real-world coordinates which are stored in meters. These real-world coordinates are labeled $x$, $y$, and $z$. Additionally, the 2D projection and depth of the joint are stored in image coordinates and meters for the depth. The 2D projection is labeled $u$ and $v$ and the depth is labeled $d$. For the final model only one dimensionallity is chosen. However, to ensure that not data is lost and that the different models can be trained on different modalities, we store all the data.

The error label is an integer which is the error id specific for joint and skeleton errors. The errors corresponding to the error ids are explained in section \ref{sec:data_labeling}. Finally, we store the timestamp of each frame. This may be useful for future applications.

\subsection{Data labeling}
\label{sec:data_labeling}

A large part of the data preparation is the labelling of the data. The data is labelled with error labels. Two areas can be labelled as erroneous. First, there are skeleton errors that occur when the pose estimator detects a human in places where there are no humans. Second, there are joint errors that occur when the pose estimator detects a joint in the wrong place. 

For example, the estimator might label the left foot as the right foot. This is a common error, especially when the body parts are close to each other. An estimator might also not detect a joint at all. This might be caused by occlusion, be it by another joint, an object, or by the image border. Most applications avoid the last cause for occlusion by defining a minimum distance from the camera and specific camera placement to ensure that the user is always fully in view.

In the data, no error is denoted with $0$. If a joint is not detected at all, it is labelled with $1$. If a joint is detected in the wrong place that is outside of the body, or somewhere where there should not be a joint, it is labelled with $2$. If a joint is detected in the approximate position of where another joint should be then it is labelled with $3$. If the whole skeleton is in the wrong position it can be labelled as faulty and subsequently, every joint will be labelled with $2$.

Implicitly, this creates two simpler general labels, either a joint is faulty, i.e. the error label is $1$, $2$, or $3$, or it is not faulty, i.e. the error label is $0$.
\section{Problem Sets}
\label{sec:problem_set}

When considering the problem of error detection in human pose estimation a distinction can be made, as to what is considered an error. While the data is labelled in a way that allows for the extraction of joint-level errors, such a fine-grain application is rarely necessary. In addition to this, joint-level error detection creates a very large search space. This may prove difficult when developing a model for such a task. Therefore, problem sets, which create different levels of abstraction of the area that is considered are defined.

The problem sets are defined by the number of objects that are considered and are defined as erroneous. There are four different problem sets. 

\begin{enumerate}[label=(\Alph*)]
  \item Joint problem set - each joint is considered as a single object
  \item Body Part problem set - each body part is considered, i.e. the individual arms, legs, torso, and head
  \item Half Body problem set - the upper and the lower body are the areas that are considered
  \item Full Body problem set - the whole body is considered as a single area
\end{enumerate}

To determine the threshold at which each area is considered faulty, the distribution of joints with errors was calculated for each of the areas. Using this distribution the threshold was picked at the 50th percentile, i.e. $50\%$ of all cases for a particular area in a problem set containing more than that number of errors. For example, in the case of the Full Body problem set, the chosen threshold is 2, i.e. in $50\%$ of all recorded frames there are two or fewer erroneous joints in the whole body. 

It was found that when considering the Full Body as an area, the body is considered faulty if more than two joints in the pose are faulty. When considering the lower and upper body, one or more and more than two faulty joints are needed for the area to be considered faulty. For the body parts to be considered faulty one joint within the specific part needs to be faulty, except for the right and left leg, where two joints need to be faulty.

The different problem sets are visualised in Figure \ref{fig:ps}.

\begin{figure}[ht]
  \centering
  \includegraphics[width=\textwidth]{figures/HPE/problem_sets.png}
  \caption[Visualisation of the Problemsets]{The visualisation of the different problem sets, (a) Joint Problem Set, (b) Body Part Problem Set, (c) Half Body Problem Set, and (d) Full Bod Problem Set.}
  \label{fig:ps}
\end{figure}

\section{Recording Setup}

Multiple iterations of the recording process were run to find the best possible setup, which reduces the light interference as much as possible and which offers the best results with the resources at hand. The camera setup that is used by SilverFit was reproduced. At SilverFit the camera is mounted at $175cm$ above the floor. The camera that is used has an accelerometer which was used to adjust the camera angle relative to the ground. The camera is angled downward at a $70^\circ$ angle. To record the dataset an Intel Realsense L515 camera was used. Additionally, for the preliminary analysis of errors an Orbbec Astra+ and a Microsoft Kinect v2 were used.

\section{Dataset Analysis}
\label{sec:dataset}

To get a better understanding of the data and how it can be used we conducted a statistical analysis. Especially the distribution of errors within the dataset is important to understand the balance of the dataset and to understand the possible outcomes of our proposed models for error detection in human pose estimation.

\subsection{Distribution of Errors}

An important aspect of the dataset is the structure and distribution of data and their labels. In total, all 13 exercises mentioned in Section \ref{sec:exercises} were recorded twice. Each recording session consists of exactly 300 frames, resulting in a total of 7800 frames. Of these, every 10th frame was labelled for a total of 780 labelled frames, which were used for the preliminary model development.

When multiple persons are detected one person might be incorrectly detected in the background. While analysing the data the person that is not labelled as faulty is selected whenever possible. If a person is labelled as faulty, each joint is marked as in an unrealistic position.

An important factor in how well a model for HPE error detection can be trained on data is the balance of the dataset. In this case, the dataset is balanced by the error labels. 

\subsubsection{Full Body Error Distribution}

In Figure \ref{fig:fb_pie} the error distribution of the Full body problem set can be seen. It can be observed that the dataset is not very balanced.

\begin{figure}[ht]
  \centering
  \includegraphics[width=0.5\textwidth]{figures/Data/dist_full_body/Error_Distribution.png}
  \caption[Error Distribution of the Full Body]{The distribution of Errors of the Full Body problem set. Of the 780 labelled frames, 308 are erroneous.}
  \label{fig:fb_pie}
\end{figure}

\subsubsection{Half Body Error Distribution}

Figure \ref{fig:hb_pie} shows a discrepancy between the error distribution of the lower body ($65.4\%$ Errors) and the upper body ($32.4\%$ Errors). The upper body is generally more stable and less error-prone than the lower body.

\begin{figure}[ht]
  \centering
  \includegraphics[width=0.7\textwidth]{figures/Data/dist_half_body/Error_Distribution_by_Body_Half.png}
  \caption[Error Distribution by Body Half]{The distribution of Errors of the Half Body problem set. Of the 780 labelled frames, 511 and 311 are erroneous for the lower and upper body respectively.}
  \label{fig:hb_pie}
\end{figure}

\subsubsection{Body part Error Distribution}

The error distribution of each body part is shown in Figure \ref{fig:lb_pie}. It can be observed that the errors of the body parts individually are very unbalanced. The left arm is the most error-prone body part with $24.1\%$ of the joints being faulty. The torso is the least error-prone body part with $10.3\%$ of the joints being faulty.

\begin{figure}[ht]
  \centering
  \includegraphics[width=0.8\textwidth]{figures/Data/dist_limbs/Error_Distribution_by_Limb.png}
  \caption[Error Distribution by Body part]{The distribution of errors of the Body Part problem set. The Left Arm is the most erroneous body part (255 errors), followed by the right arm (178 errors), the right leg (176 errors), the left leg (174 errors), and the head (119 errors). The least errors occur in the torso (80 errors in 780 frames).}
  \label{fig:lb_pie}
\end{figure}

\subsubsection{Joint Error Distribution}

Figure \ref{fig:jt_pie} shows that the major part of the errors that occur are errors with Label 2, i.e. the joint is detected at the wrong place. The second most common error is Label 1, i.e. the joint is not detected at all. The least common error is Label 3, i.e. the joint is detected in the approximate position of where another joint should be.

\begin{figure}[ht]
  \centering
  \includegraphics[width=0.5\textwidth]{figures/Data/dist_joints/Error_Distribution.png}
  \caption[Error Distribution for each error class]{The distribution of each error class. The Right Ankle is the most erroneous joint (542 errors), followed by the left ankle (520 errors), the left hand (249 errors), the left wrist (183 errors), the right hand (169 errors), the right wrist (146 errors), the left hip (136 errors), the right hip (122 errors), the right knee (147 errors), the waist (117 errors), the left knee (127 errors), the right elbow (101 errors), the left elbow (77 errors), the head (75 errors), the neck (73 errors), the right shoulder (65 errors), the right collar (63 errors), the torso (63 errors), and the left collar (63 errors). The least errors occur in the left shoulder (38 errors in 780 frames).}
  \label{fig:jt_pie}
\end{figure}

